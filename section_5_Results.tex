\section{Results and discussion}
\label{sec:experiment}

\subsection{Impact of stochastic decomposition}
To assess the impact of the stochastic decomposition method on PMAO, we create four variants of PMAO listed in Table~\ref{tab:variants}. Here we consider increasing the iteration so that the intensify the effect of stochastic decomposition. In Table~\ref{tab:pmao-variants-a},  \ref{tab:pmao-variants-b} we report the best FN rate among the 30 solutions generated by these variants for datasets under set A and B respectively. To assist in comparing the performance of four variants on each dataset, we mark the lower (i.e., better) FN rate values with darker shade. We see that the columns denoted by 3I-S and 8I-S contains more darker shades than 3I-D and 8I-D which is an indication of the positive impact of stochastic decomposition on the performance of PMAO. To contrast among these variants, we apply the Friedman Aligned Ranks test~\cite{hodges2012rank} followed by complementary Holm’s post-hoc procedure~\cite{holm1979simple} on Table~\ref{tab:pmao-variants-a},  \ref{tab:pmao-variants-b}, as recommended by~\cite{derrac2011practical} using 95\% confidence level. The results the summarized in Table~\ref{tab:test-pmao-variants}. The lower ranks of stochastic variants and significant difference between 3I-D and 8I-S clearly shows the effectiveness of stochastic decomposition. Note that, PASTA uses 3 iteration by default and most of its improvement is achieved in the first iteration. Moreover, stochastic decomposition makes sense in the context of MO principles as discussed earlier in Section~\ref{subsec:stocastic}. We verified this by conducting similar analysis (Supplementary Tables~\ref{tab:pasta-variants-a}, \ref{tab:pasta-variants-b},  \ref{tab:test-pasta-variants}) based on four PASTA variants. We found no significant difference between any pair as expected and 8I-D seemed to be the best variant. In the subsequent sections, we use the 8I-S variant of PMAO and the 8I-D variant of PASTA to ensure level playing ground.

\begin{table}[!htbp]
	\small
	\caption{PMAO variants based on iteration count and guide tree decomposition strategy. }
	\begin{tabular}{l|r|l}
		Variant & Iteration & Tree decomposition\\
		\hline
		3I-D  & 3     & Default (\textit{mincluster}) \\
		\hline
		8I-D  & 8     & Default (\textit{mincluster}) \\
		\hline
		3I-S  & 3     & Stochastic \\
		\hline
		8I-S  & 8     & Stochastic\\
	\end{tabular}%
	\label{tab:variants}%
\end{table}%

% Table generated by Excel2LaTeX from sheet 'Sheet4'
\begin{table}[!htbp]
	%\centering
	\caption{Best FN rate achieved by the four variants of PMAO for each dataset in set A. On each row, the lower (better) FN rates are marked with darker shade.}
	\begin{tabular}{|l|r|r|r|r|}
		\hline
		\multirow{2}{*}{Dataset} & \multicolumn{4}{c|}{\makecell{Best FN rate achieved\\by PMAO variants}} \\
		\cline{2-5}          & 3I-D & 8I-D & 3I-S & 8I-S \\
		\hline
		BB11005 & \cellcolor[rgb]{ .988,  1,  .992}0.18 & \cellcolor[rgb]{ .384,  .745,  .478}0.09 & \cellcolor[rgb]{ .384,  .745,  .478}0.09 & \cellcolor[rgb]{ .384,  .745,  .478}0.09 \\
		\hline
		BB11018 & \cellcolor[rgb]{ .988,  1,  .992}0.27 & \cellcolor[rgb]{ .384,  .745,  .478}0.18 & \cellcolor[rgb]{ .384,  .745,  .478}0.18 & \cellcolor[rgb]{ .384,  .745,  .478}0.18 \\
		\hline
		BB11033 & \cellcolor[rgb]{ .988,  1,  .992}0.38 & \cellcolor[rgb]{ .988,  1,  .992}0.38 & \cellcolor[rgb]{ .988,  1,  .992}0.38 & \cellcolor[rgb]{ .988,  1,  .992}0.38 \\
		\hline
		BB11020 & \cellcolor[rgb]{ .988,  1,  .992}0.33 & \cellcolor[rgb]{ .988,  1,  .992}0.33 & \cellcolor[rgb]{ .988,  1,  .992}0.33 & \cellcolor[rgb]{ .988,  1,  .992}0.33 \\
		\hline
		BB12001 & \cellcolor[rgb]{ .988,  1,  .992}0.13 & \cellcolor[rgb]{ .988,  1,  .992}0.13 & \cellcolor[rgb]{ .988,  1,  .992}0.13 & \cellcolor[rgb]{ .988,  1,  .992}0.13 \\
		\hline
		BB12013 & \cellcolor[rgb]{ .384,  .745,  .478}0.00 & \cellcolor[rgb]{ .988,  1,  .992}0.20 & \cellcolor[rgb]{ .384,  .745,  .478}0.00 & \cellcolor[rgb]{ .988,  1,  .992}0.20 \\
		\hline
		BB12022 & \cellcolor[rgb]{ .988,  1,  .992}0.00 & \cellcolor[rgb]{ .988,  1,  .992}0.00 & \cellcolor[rgb]{ .988,  1,  .992}0.00 & \cellcolor[rgb]{ .988,  1,  .992}0.00 \\
		\hline
		BB12035 & \cellcolor[rgb]{ .988,  1,  .992}0.04 & \cellcolor[rgb]{ .384,  .745,  .478}0.00 & \cellcolor[rgb]{ .988,  1,  .992}0.04 & \cellcolor[rgb]{ .988,  1,  .992}0.04 \\
		\hline
		BB12044 & \cellcolor[rgb]{ .988,  1,  .992}0.38 & \cellcolor[rgb]{ .988,  1,  .992}0.38 & \cellcolor[rgb]{ .988,  1,  .992}0.38 & \cellcolor[rgb]{ .988,  1,  .992}0.38 \\
		\hline
		BB20001 & \cellcolor[rgb]{ .384,  .745,  .478}0.23 & \cellcolor[rgb]{ .988,  1,  .992}0.46 & \cellcolor[rgb]{ .384,  .745,  .478}0.23 & \cellcolor[rgb]{ .384,  .745,  .478}0.23 \\
		\hline
		BB20010 & \cellcolor[rgb]{ .988,  1,  .992}0.31 & \cellcolor[rgb]{ .988,  1,  .992}0.31 & \cellcolor[rgb]{ .384,  .745,  .478}0.08 & \cellcolor[rgb]{ .384,  .745,  .478}0.08 \\
		\hline
		BB20022 & \cellcolor[rgb]{ .988,  1,  .992}0.11 & \cellcolor[rgb]{ .988,  1,  .992}0.11 & \cellcolor[rgb]{ .988,  1,  .992}0.11 & \cellcolor[rgb]{ .384,  .745,  .478}0.09 \\
		\hline
		BB20033 & \cellcolor[rgb]{ .988,  1,  .992}0.36 & \cellcolor[rgb]{ .988,  1,  .992}0.36 & \cellcolor[rgb]{ .988,  1,  .992}0.36 & \cellcolor[rgb]{ .384,  .745,  .478}0.24 \\
		\hline
		BB20041 & \cellcolor[rgb]{ .988,  1,  .992}0.33 & \cellcolor[rgb]{ .584,  .827,  .647}0.29 & \cellcolor[rgb]{ .384,  .745,  .478}0.27 & \cellcolor[rgb]{ .988,  1,  .992}0.33 \\
		\hline
		BB30002 & \cellcolor[rgb]{ .988,  1,  .992}0.32 & \cellcolor[rgb]{ .988,  1,  .992}0.32 & \cellcolor[rgb]{ .384,  .745,  .478}0.14 & \cellcolor[rgb]{ .502,  .792,  .58}0.18 \\
		\hline
		BB30008 & \cellcolor[rgb]{ .533,  .808,  .604}0.24 & \cellcolor[rgb]{ .988,  1,  .992}0.33 & \cellcolor[rgb]{ .835,  .933,  .863}0.30 & \cellcolor[rgb]{ .384,  .745,  .478}0.21 \\
		\hline
		BB30015 & \cellcolor[rgb]{ .988,  1,  .992}0.17 & \cellcolor[rgb]{ .988,  1,  .992}0.17 & \cellcolor[rgb]{ .988,  1,  .992}0.17 & \cellcolor[rgb]{ .988,  1,  .992}0.17 \\
		\hline
		BB30022 & \cellcolor[rgb]{ .682,  .871,  .733}0.48 & \cellcolor[rgb]{ .682,  .871,  .733}0.48 & \cellcolor[rgb]{ .988,  1,  .992}0.49 & \cellcolor[rgb]{ .384,  .745,  .478}0.46 \\
		\hline
		BB40001 & \cellcolor[rgb]{ .988,  1,  .992}0.48 & \cellcolor[rgb]{ .784,  .914,  .82}0.44 & \cellcolor[rgb]{ .988,  1,  .992}0.48 & \cellcolor[rgb]{ .384,  .745,  .478}0.36 \\
		\hline
		BB40013 & \cellcolor[rgb]{ .988,  1,  .992}0.31 & \cellcolor[rgb]{ .988,  1,  .992}0.31 & \cellcolor[rgb]{ .988,  1,  .992}0.31 & \cellcolor[rgb]{ .384,  .745,  .478}0.25 \\
		\hline
		BB40025 & \cellcolor[rgb]{ .988,  1,  .992}0.00 & \cellcolor[rgb]{ .988,  1,  .992}0.00 & \cellcolor[rgb]{ .988,  1,  .992}0.00 & \cellcolor[rgb]{ .988,  1,  .992}0.00 \\
		\hline
		BB40038 & \cellcolor[rgb]{ .988,  1,  .992}0.10 & \cellcolor[rgb]{ .988,  1,  .992}0.10 & \cellcolor[rgb]{ .988,  1,  .992}0.10 & \cellcolor[rgb]{ .988,  1,  .992}0.10 \\
		\hline
		BB40048 & \cellcolor[rgb]{ .988,  1,  .992}0.29 & \cellcolor[rgb]{ .988,  1,  .992}0.29 & \cellcolor[rgb]{ .384,  .745,  .478}0.21 & \cellcolor[rgb]{ .988,  1,  .992}0.29 \\
		\hline
		BB50001 & \cellcolor[rgb]{ .988,  1,  .992}0.29 & \cellcolor[rgb]{ .988,  1,  .992}0.29 & \cellcolor[rgb]{ .988,  1,  .992}0.29 & \cellcolor[rgb]{ .988,  1,  .992}0.29 \\
		\hline
		BB50005 & \cellcolor[rgb]{ .686,  .871,  .733}0.25 & \cellcolor[rgb]{ .988,  1,  .992}0.38 & \cellcolor[rgb]{ .384,  .745,  .478}0.13 & \cellcolor[rgb]{ .686,  .871,  .733}0.25 \\
		\hline
		BB50010 & \cellcolor[rgb]{ .988,  1,  .992}0.07 & \cellcolor[rgb]{ .384,  .745,  .478}0.00 & \cellcolor[rgb]{ .384,  .745,  .478}0.00 & \cellcolor[rgb]{ .384,  .745,  .478}0.00 \\
		\hline
		BB50016 & \cellcolor[rgb]{ .682,  .871,  .733}0.13 & \cellcolor[rgb]{ .988,  1,  .992}0.20 & \cellcolor[rgb]{ .384,  .745,  .478}0.07 & \cellcolor[rgb]{ .384,  .745,  .478}0.07 \\
		\hline
	\end{tabular}%
	\label{tab:pmao-variants-a}%
\end{table}%

% Table generated by Excel2LaTeX from sheet 'Sheet4'
\begin{table}[!htbp]
	\small
	%\centering
	\caption{Best FN rate achieved by the four variants of PMAO for each dataset in set B. On each row, the lower (better) FN rates are marked with darker shade.}
	\begin{tabular}{|l|r|r|r|r|}
		\hline
		\multirow{2}{*}{Dataset} & \multicolumn{4}{c|}{\makecell{Best FN rate achieved\\by PMAO variants}} \\
		\cline{2-5}          & \multicolumn{1}{l|}{3I-D} & \multicolumn{1}{l|}{8I-D} & \multicolumn{1}{l|}{3I-S} & \multicolumn{1}{l|}{8I-S} \\
		\hline
		BB11007 & \cellcolor[rgb]{ .384,  .745,  .478}0.33 & \cellcolor[rgb]{ .384,  .745,  .478}0.33 & \cellcolor[rgb]{ .988,  1,  .992}0.50 & \cellcolor[rgb]{ .384,  .745,  .478}0.33 \\
		\hline
		BB11034 & \cellcolor[rgb]{ .988,  1,  .992}0.20 & \cellcolor[rgb]{ .384,  .745,  .478}0.00 & \cellcolor[rgb]{ .988,  1,  .992}0.20 & \cellcolor[rgb]{ .988,  1,  .992}0.20 \\
		\hline
		BB11038 & \cellcolor[rgb]{ .384,  .745,  .478}0.00 & \cellcolor[rgb]{ .384,  .745,  .478}0.00 & \cellcolor[rgb]{ .988,  1,  .992}0.40 & \cellcolor[rgb]{ .384,  .745,  .478}0.00 \\
		\hline
		BB11019 & \cellcolor[rgb]{ .988,  1,  .992}0.14 & \cellcolor[rgb]{ .988,  1,  .992}0.14 & \cellcolor[rgb]{ .988,  1,  .992}0.14 & \cellcolor[rgb]{ .988,  1,  .992}0.14 \\
		\hline
		BB12005 & \cellcolor[rgb]{ .988,  1,  .992}0.33 & \cellcolor[rgb]{ .988,  1,  .992}0.33 & \cellcolor[rgb]{ .988,  1,  .992}0.33 & \cellcolor[rgb]{ .988,  1,  .992}0.33 \\
		\hline
		BB12029 & \cellcolor[rgb]{ .988,  1,  .992}0.44 & \cellcolor[rgb]{ .988,  1,  .992}0.44 & \cellcolor[rgb]{ .384,  .745,  .478}0.33 & \cellcolor[rgb]{ .384,  .745,  .478}0.33 \\
		\hline
		BB12026 & \cellcolor[rgb]{ .384,  .745,  .478}0.27 & \cellcolor[rgb]{ .384,  .745,  .478}0.27 & \cellcolor[rgb]{ .384,  .745,  .478}0.27 & \cellcolor[rgb]{ .988,  1,  .992}0.33 \\
		\hline
		BB12037 & \cellcolor[rgb]{ .988,  1,  .992}0.10 & \cellcolor[rgb]{ .988,  1,  .992}0.10 & \cellcolor[rgb]{ .988,  1,  .992}0.10 & \cellcolor[rgb]{ .988,  1,  .992}0.10 \\
		\hline
		BB20002 & \cellcolor[rgb]{ .686,  .871,  .733}0.53 & \cellcolor[rgb]{ .384,  .745,  .478}0.47 & \cellcolor[rgb]{ .988,  1,  .992}0.59 & \cellcolor[rgb]{ .686,  .871,  .733}0.53 \\
		\hline
		BB20012 & \cellcolor[rgb]{ .988,  1,  .992}0.29 & \cellcolor[rgb]{ .745,  .894,  .784}0.21 & \cellcolor[rgb]{ .745,  .894,  .784}0.21 & \cellcolor[rgb]{ .384,  .745,  .478}0.08 \\
		\hline
		BB20030 & \cellcolor[rgb]{ .988,  1,  .992}0.64 & \cellcolor[rgb]{ .384,  .745,  .478}0.50 & \cellcolor[rgb]{ .988,  1,  .992}0.64 & \cellcolor[rgb]{ .988,  1,  .992}0.64 \\
		\hline
		BB20037 & \cellcolor[rgb]{ .988,  1,  .992}0.13 & \cellcolor[rgb]{ .988,  1,  .992}0.13 & \cellcolor[rgb]{ .988,  1,  .992}0.13 & \cellcolor[rgb]{ .988,  1,  .992}0.13 \\
		\hline
		BB30003 & \cellcolor[rgb]{ .988,  1,  .992}0.27 & \cellcolor[rgb]{ .384,  .745,  .478}0.25 & \cellcolor[rgb]{ .682,  .871,  .733}0.26 & \cellcolor[rgb]{ .682,  .871,  .733}0.26 \\
		\hline
		BB30021 & \cellcolor[rgb]{ .729,  .89,  .769}0.30 & \cellcolor[rgb]{ .816,  .925,  .843}0.31 & \cellcolor[rgb]{ .384,  .745,  .478}0.27 & \cellcolor[rgb]{ .988,  1,  .992}0.32 \\
		\hline
		BB30026 & \cellcolor[rgb]{ .384,  .745,  .478}0.15 & \cellcolor[rgb]{ .384,  .745,  .478}0.15 & \cellcolor[rgb]{ .988,  1,  .992}0.17 & \cellcolor[rgb]{ .384,  .745,  .478}0.15 \\
		\hline
		BB30011 & \cellcolor[rgb]{ .988,  1,  .992}0.32 & \cellcolor[rgb]{ .384,  .745,  .478}0.31 & \cellcolor[rgb]{ .384,  .745,  .478}0.31 & \cellcolor[rgb]{ .384,  .745,  .478}0.31 \\
		\hline
		BB40009 & \cellcolor[rgb]{ .988,  1,  .992}0.13 & \cellcolor[rgb]{ .988,  1,  .992}0.13 & \cellcolor[rgb]{ .988,  1,  .992}0.13 & \cellcolor[rgb]{ .988,  1,  .992}0.13 \\
		\hline
		BB40019 & \cellcolor[rgb]{ .988,  1,  .992}0.29 & \cellcolor[rgb]{ .988,  1,  .992}0.29 & \cellcolor[rgb]{ .384,  .745,  .478}0.14 & \cellcolor[rgb]{ .988,  1,  .992}0.29 \\
		\hline
		BB40033 & \cellcolor[rgb]{ .988,  1,  .992}0.06 & \cellcolor[rgb]{ .988,  1,  .992}0.06 & \cellcolor[rgb]{ .988,  1,  .992}0.06 & \cellcolor[rgb]{ .988,  1,  .992}0.06 \\
		\hline
		BB40006 & \cellcolor[rgb]{ .988,  1,  .992}0.00 & \cellcolor[rgb]{ .988,  1,  .992}0.00 & \cellcolor[rgb]{ .988,  1,  .992}0.00 & \cellcolor[rgb]{ .988,  1,  .992}0.00 \\
		\hline
		BB50002 & \cellcolor[rgb]{ .384,  .745,  .478}0.40 & \cellcolor[rgb]{ .988,  1,  .992}0.50 & \cellcolor[rgb]{ .988,  1,  .992}0.50 & \cellcolor[rgb]{ .384,  .745,  .478}0.40 \\
		\hline
		BB50009 & \cellcolor[rgb]{ .988,  1,  .992}0.08 & \cellcolor[rgb]{ .988,  1,  .992}0.08 & \cellcolor[rgb]{ .988,  1,  .992}0.08 & \cellcolor[rgb]{ .988,  1,  .992}0.08 \\
		\hline
		BB50014 & \cellcolor[rgb]{ .988,  1,  .992}0.41 & \cellcolor[rgb]{ .988,  1,  .992}0.41 & \cellcolor[rgb]{ .988,  1,  .992}0.41 & \cellcolor[rgb]{ .384,  .745,  .478}0.26 \\
		\hline
		BB50006 & \cellcolor[rgb]{ .988,  1,  .992}0.16 & \cellcolor[rgb]{ .988,  1,  .992}0.16 & \cellcolor[rgb]{ .988,  1,  .992}0.16 & \cellcolor[rgb]{ .384,  .745,  .478}0.12 \\
		\hline
	\end{tabular}%
	\label{tab:pmao-variants-b}%
\end{table}%

% Table generated by Excel2LaTeX from sheet 'stat test'
\begin{table}[!htbp]
	\small
	\caption{\underline{Friedman Aligned Ranks test (Column 2):} Friedman Aligned ranks (lower is better) of the four variants of PMAO based on Table~\ref{tab:pmao-variants-a},  \ref{tab:pmao-variants-b}. We also show the computed statistics and corresponding $ p $-value. 
		\underline{Holm's post-hoc procedure (Columns 3 - 6):} Comparison among the PMAO variants using the Holm's post-hoc procedures. Each entry shows the adjusted $p$-value which indicates the significance of the difference in performance between two methods.}
	\begin{tabular}{|l|r||c|c|c|c|}
		\hline
		\multicolumn{1}{|c|}{1} & \multicolumn{1}{c||}{2} & \multicolumn{1}{c|}{3} & \multicolumn{1}{c|}{4} & \multicolumn{1}{c|}{5} & 6 \\
		\hline
		\multirow{2}{*}{\makecell{PMAO\\ variants}} & \multirow{2}{*}{\makecell{Friedman\\Aligned rank*}} & \multicolumn{4}{c|}{Holm's adjusted $p$-value} \\
		\cline{3-6}          &       & 8I-S & 3I-S & 8I-D & 3I-D \\
		\hline
		8I-S & 82.3137 & \multicolumn{1}{c|}{-} & \multicolumn{1}{r|}{0.4032} & \multicolumn{1}{r|}{0.1431} & \multicolumn{1}{r|}{\cellcolor[rgb]{ .384,  .745,  .478}0.0077} \\
		\hline
		3I-S & 99.8137 & \multicolumn{1}{r|}{0.4032} & \multicolumn{1}{c|}{-} & \multicolumn{1}{r|}{0.6038} & \multicolumn{1}{r|}{0.3387} \\
		\hline
		8I-D & 107.9020 & \multicolumn{1}{r|}{0.1431} & \multicolumn{1}{r|}{0.6038} & \multicolumn{1}{c|}{-} & \multicolumn{1}{r|}{0.6038} \\
		\hline
		3I-D & 119.9706 & \multicolumn{1}{r|}{\cellcolor[rgb]{ .384,  .745,  .478}0.0077} & \multicolumn{1}{r|}{0.3387} & \multicolumn{1}{r|}{0.6038} & - \\
		\hline
		*Statistic & 9.0393 & \multicolumn{4}{c|}{\multirow{2}{*}{N/A}} \\
		\cline{1-2}    *$p$-value & 0.0288 & \multicolumn{4}{c|}{} \\
		\hline
	\end{tabular}%
	\label{tab:test-pmao-variants}%
\end{table}%
\subsection{Comparison between PASTA and PMAO}
We compare the solutions generated by PMAO (8I-S variant) with the output of PASTA (8I-D variant) in terms of FN rate. So we show the PASTA FN rate  alongside the best FN rate achieved by PMAO in Table~\ref{tab:pmao-pasta-a} and \ref{tab:pmao-pasta-b} where the better values are marked with darker shades. Furthermore, to give an overall picture of the tree-space generated by PMAO we include the average FN rate and count of those PMAO solutions that are equivalent or better than PASTA. We find that the best FN rates of PMAO are clearly ahead of PASTA's FN rate. In several cases the improvement is more than 50\% (e.g., BB11020, Bb20001, BB20010, BB40028, BB50016, BB11038, BB12037, BB20012, BB40033, BB50009, etc.). Note that there are several solutions better than PASTA in the tree-space. In 39 cases out of 51, there are more than 10 solution (among the 30 alternatives) better than PASTA. There results demonstrates the advantages of PMAO over PASTA. These results for all 218 datasets are available in Supplementary Tables~\ref{tab:pmao-pasta-rv11}, \ref{tab:pmao-pasta-rv12}, \ref{tab:pmao-pasta-rv20}, \ref{tab:pmao-pasta-rv30}, \ref{tab:pmao-pasta-rv40}, \ref{tab:pmao-pasta-rv50}.
% Table generated by Excel2LaTeX from sheet 'stat-8I-30w-S'
\begin{table}[!htbp]
	\small
	%\centering
	\caption{Comparison of the 30 solutions generated by PMAO with respect to PASTA in terms of FN rate on set A datasets. For PMAO, we show the best FN rate along with the average FN rate 
		and count of its solutions better or equivalent to PASTA. The better values are marked with darker shade.}
	\begin{tabular}{|l|r|r|r||r|}
		\hline
		\multirow{2}{*}{Dataset} & \multirow{2}{*}{\makecell{PASTA\\FN rate}} & \multicolumn{3}{c|}{\makecell{PMAO solutions better \\or equal quality to PASTA}} \\
		\cline{3-5}          &       & \multicolumn{1}{l|}{Best FN} & \multicolumn{1}{l|}{Avg FN} & \multicolumn{1}{l|}{Count} \\
		\hline
		BB11005 & \cellcolor[rgb]{ .988,  .988,  1}0.55 & \cellcolor[rgb]{ .388,  .745,  .482}0.09 & \cellcolor[rgb]{ .753,  .894,  .8}0.37 & \cellcolor[rgb]{ .976,  .451,  .459}28 \\
		\hline
		BB11018 & \cellcolor[rgb]{ .988,  .988,  1}0.27 & \cellcolor[rgb]{ .388,  .745,  .482}0.18 & \cellcolor[rgb]{ .784,  .906,  .824}0.24 & \cellcolor[rgb]{ .988,  .875,  .886}6 \\
		\hline
		BB11033 & \cellcolor[rgb]{ .988,  .988,  1}0.38 & \cellcolor[rgb]{ .988,  .988,  1}0.38 & \cellcolor[rgb]{ .988,  .988,  1}0.38 & \cellcolor[rgb]{ .988,  .894,  .906}5 \\
		\hline
		BB11020 & \cellcolor[rgb]{ .988,  .988,  1}0.83 & \cellcolor[rgb]{ .388,  .745,  .482}0.33 & \cellcolor[rgb]{ .733,  .882,  .78}0.62 & \cellcolor[rgb]{ .973,  .412,  .42}30 \\
		\hline
		BB12001 & \cellcolor[rgb]{ .988,  .988,  1}0.25 & \cellcolor[rgb]{ .388,  .745,  .482}0.13 & \cellcolor[rgb]{ .906,  .953,  .929}0.23 & \cellcolor[rgb]{ .98,  .702,  .71}15 \\
		\hline
		BB12013 & \cellcolor[rgb]{ .988,  .988,  1}0.20 & \cellcolor[rgb]{ .988,  .988,  1}0.20 & \cellcolor[rgb]{ .988,  .988,  1}0.20 & \cellcolor[rgb]{ .973,  .412,  .42}30 \\
		\hline
		BB12022 & \cellcolor[rgb]{ .988,  .988,  1}0.00 & \cellcolor[rgb]{ .988,  .988,  1}0.00 & \cellcolor[rgb]{ .988,  .988,  1}0.00 & \cellcolor[rgb]{ .976,  .51,  .518}25 \\
		\hline
		BB12035 & \cellcolor[rgb]{ .388,  .745,  .482}0.00  &   0.04    &   N.A.    & \cellcolor[rgb]{ .988,  .988,  1}0 \\
		\hline
		BB12044 & \cellcolor[rgb]{ .988,  .988,  1}0.50 & \cellcolor[rgb]{ .388,  .745,  .482}0.38 & \cellcolor[rgb]{ .706,  .875,  .757}0.44 & \cellcolor[rgb]{ .973,  .412,  .42}30 \\
		\hline
		BB20001 & \cellcolor[rgb]{ .988,  .988,  1}0.54 & \cellcolor[rgb]{ .388,  .745,  .482}0.23 & \cellcolor[rgb]{ .843,  .929,  .875}0.46 & \cellcolor[rgb]{ .976,  .51,  .518}25 \\
		\hline
		BB20010 & \cellcolor[rgb]{ .988,  .988,  1}0.35 & \cellcolor[rgb]{ .388,  .745,  .482}0.08 & \cellcolor[rgb]{ .871,  .941,  .898}0.29 & \cellcolor[rgb]{ .976,  .549,  .557}23 \\
		\hline
		BB20022 & \cellcolor[rgb]{ .988,  .988,  1}0.11 & \cellcolor[rgb]{ .388,  .745,  .482}0.09 & \cellcolor[rgb]{ .925,  .961,  .945}0.11 & \cellcolor[rgb]{ .984,  .796,  .808}10 \\
		\hline
		BB20033 & \cellcolor[rgb]{ .988,  .988,  1}0.36 & \cellcolor[rgb]{ .388,  .745,  .482}0.24 & \cellcolor[rgb]{ .784,  .906,  .824}0.32 & \cellcolor[rgb]{ .984,  .816,  .827}9 \\
		\hline
		BB20041 & \cellcolor[rgb]{ .988,  .988,  1}0.38 & \cellcolor[rgb]{ .388,  .745,  .482}0.33 & \cellcolor[rgb]{ .8,  .91,  .835}0.36 & \cellcolor[rgb]{ .984,  .835,  .847}8 \\
		\hline
		BB30002 & \cellcolor[rgb]{ .988,  .988,  1}0.32 & \cellcolor[rgb]{ .388,  .745,  .482}0.18 & \cellcolor[rgb]{ .847,  .929,  .878}0.29 & \cellcolor[rgb]{ .98,  .702,  .71}15 \\
		\hline
		BB30008 & \cellcolor[rgb]{ .988,  .988,  1}0.33 & \cellcolor[rgb]{ .388,  .745,  .482}0.21 & \cellcolor[rgb]{ .878,  .941,  .906}0.31 & \cellcolor[rgb]{ .984,  .722,  .729}14 \\
		\hline
		BB30015 & \cellcolor[rgb]{ .988,  .988,  1}0.17 & \cellcolor[rgb]{ .988,  .988,  1}0.17 & \cellcolor[rgb]{ .988,  .988,  1}0.17 & \cellcolor[rgb]{ .984,  .761,  .769}12 \\
		\hline
		BB30022 & \cellcolor[rgb]{ .988,  .988,  1}0.51 & \cellcolor[rgb]{ .388,  .745,  .482}0.46 & \cellcolor[rgb]{ .902,  .953,  .925}0.50 & \cellcolor[rgb]{ .98,  .663,  .675}17 \\
		\hline
		BB40001 & \cellcolor[rgb]{ .988,  .988,  1}0.48 & \cellcolor[rgb]{ .388,  .745,  .482}0.36 & \cellcolor[rgb]{ .745,  .89,  .792}0.43 & \cellcolor[rgb]{ .984,  .796,  .808}10 \\
		\hline
		BB40013 & \cellcolor[rgb]{ .988,  .988,  1}0.38 & \cellcolor[rgb]{ .388,  .745,  .482}0.25 & \cellcolor[rgb]{ .753,  .89,  .796}0.33 & \cellcolor[rgb]{ .98,  .682,  .694}16 \\
		\hline
		BB40025 & \cellcolor[rgb]{ .988,  .988,  1}0.00 & \cellcolor[rgb]{ .988,  .988,  1}0.00 & \cellcolor[rgb]{ .988,  .988,  1}0.00 & \cellcolor[rgb]{ .98,  .569,  .576}22 \\
		\hline
		BB40038 & \cellcolor[rgb]{ .988,  .988,  1}0.25 & \cellcolor[rgb]{ .388,  .745,  .482}0.10 & \cellcolor[rgb]{ .788,  .906,  .827}0.20 & \cellcolor[rgb]{ .98,  .624,  .635}19 \\
		\hline
		BB40048 & \cellcolor[rgb]{ .988,  .988,  1}0.43 & \cellcolor[rgb]{ .388,  .745,  .482}0.29 & \cellcolor[rgb]{ .447,  .769,  .533}0.30 & \cellcolor[rgb]{ .973,  .412,  .42}30 \\
		\hline
		BB50001 & \cellcolor[rgb]{ .988,  .988,  1}0.29 & \cellcolor[rgb]{ .988,  .988,  1}0.29 & \cellcolor[rgb]{ .988,  .988,  1}0.29 & \cellcolor[rgb]{ .98,  .663,  .675}17 \\
		\hline
		BB50005 & \cellcolor[rgb]{ .988,  .988,  1}0.38 & \cellcolor[rgb]{ .388,  .745,  .482}0.25 & \cellcolor[rgb]{ .827,  .922,  .859}0.34 & \cellcolor[rgb]{ .973,  .412,  .42}30 \\
		\hline
		BB50010 & \cellcolor[rgb]{ .988,  .988,  1}0.00 & \cellcolor[rgb]{ .988,  .988,  1}0.00 & \cellcolor[rgb]{ .988,  .988,  1}0.00 & \cellcolor[rgb]{ .988,  .855,  .867}7 \\
		\hline
		BB50016 & \cellcolor[rgb]{ .988,  .988,  1}0.47 & \cellcolor[rgb]{ .388,  .745,  .482}0.07 & \cellcolor[rgb]{ .584,  .824,  .651}0.20 & \cellcolor[rgb]{ .976,  .451,  .459}28 \\
		\hline
	\end{tabular}%
	\label{tab:pmao-pasta-a}%
\end{table}%

% Table generated by Excel2LaTeX from sheet 'stat-8I-30w-S'
\begin{table}[!htbp]
	\small
	\caption{Comparison of the 30 solutions generated by PMAO with respect to PASTA in terms of FN rate on set B datasets. For PMAO, we show the best FN rate along with the average FN rate 
		and count of its solutions better or equivalent to PASTA. The better values are marked with darker shade.}
	\begin{tabular}{|l|r|r|r||r|}
		\hline
		\multirow{2}{*}{Dataset} & \multirow{2}{*}{\makecell{PASTA\\FN rate}} & \multicolumn{3}{c|}{\makecell{PMAO solutions better \\or equal quality to PASTA}} \\
		\cline{3-5}          &       & \multicolumn{1}{l|}{Best FN} & \multicolumn{1}{l|}{Avg FN} & \multicolumn{1}{l|}{Count} \\
		\hline
		BB11007 & \cellcolor[rgb]{ .988,  1,  .992}0.50 & \cellcolor[rgb]{ .384,  .745,  .478}0.33 & \cellcolor[rgb]{ .906,  .965,  .922}0.48 & \cellcolor[rgb]{ .984,  .714,  .722}15 \\
		\hline
		BB11034 & \cellcolor[rgb]{ .988,  1,  .992}0.40 & \cellcolor[rgb]{ .384,  .745,  .478}0.20 & \cellcolor[rgb]{ .753,  .898,  .792}0.32 & \cellcolor[rgb]{ .98,  .651,  .663}18 \\
		\hline
		BB11038 & \cellcolor[rgb]{ .988,  1,  .992}0.40 & \cellcolor[rgb]{ .384,  .745,  .478}0.00 & \cellcolor[rgb]{ .937,  .976,  .949}0.37 & \cellcolor[rgb]{ .984,  .773,  .78}12 \\
		\hline
		BB11019 & \cellcolor[rgb]{ .988,  1,  .992}0.29 & \cellcolor[rgb]{ .384,  .745,  .478}0.14 & \cellcolor[rgb]{ .918,  .969,  .933}0.27 & \cellcolor[rgb]{ .976,  .475,  .482}27 \\
		\hline
		BB12005 & \cellcolor[rgb]{ .988,  1,  .992}0.33 & \cellcolor[rgb]{ .988,  1,  .992}0.33 & \cellcolor[rgb]{ .988,  1,  .992}0.33 & \cellcolor[rgb]{ .976,  .455,  .463}28 \\
		\hline
		BB12029 & \cellcolor[rgb]{ .988,  1,  .992}0.44 & \cellcolor[rgb]{ .384,  .745,  .478}0.33 & \cellcolor[rgb]{ .863,  .945,  .882}0.42 & \cellcolor[rgb]{ .976,  .435,  .443}29 \\
		\hline
		BB12026 & \cellcolor[rgb]{ .988,  1,  .992}0.53 & \cellcolor[rgb]{ .384,  .745,  .478}0.33 & \cellcolor[rgb]{ .816,  .925,  .847}0.48 & \cellcolor[rgb]{ .984,  .753,  .761}13 \\
		\hline
		BB12037 & \cellcolor[rgb]{ .988,  1,  .992}0.40 & \cellcolor[rgb]{ .384,  .745,  .478}0.10 & \cellcolor[rgb]{ .859,  .945,  .882}0.34 & \cellcolor[rgb]{ .988,  .851,  .863}8 \\
		\hline
		BB20002 & \cellcolor[rgb]{ .988,  1,  .992}0.65 & \cellcolor[rgb]{ .384,  .745,  .478}0.53 & \cellcolor[rgb]{ .835,  .933,  .863}0.62 & \cellcolor[rgb]{ .984,  .812,  .824}10 \\
		\hline
		BB20012 & \cellcolor[rgb]{ .988,  1,  .992}0.29 & \cellcolor[rgb]{ .384,  .745,  .478}0.08 & \cellcolor[rgb]{ .851,  .941,  .875}0.25 & \cellcolor[rgb]{ .976,  .494,  .502}26 \\
		\hline
		BB20030 & \cellcolor[rgb]{ .988,  1,  .992}0.64 & \cellcolor[rgb]{ .988,  1,  .992}0.64 & \cellcolor[rgb]{ .988,  1,  .992}0.64 & \cellcolor[rgb]{ .988,  .871,  .882}7 \\
		\hline
		BB20037 & \cellcolor[rgb]{ .988,  1,  .992}0.13 & \cellcolor[rgb]{ .988,  1,  .992}0.13 & \cellcolor[rgb]{ .988,  1,  .992}0.13 & \cellcolor[rgb]{ .988,  .988,  1}1 \\
		\hline
		BB30003 & \cellcolor[rgb]{ .988,  1,  .992}0.30 & \cellcolor[rgb]{ .384,  .745,  .478}0.26 & \cellcolor[rgb]{ .773,  .906,  .808}0.29 & \cellcolor[rgb]{ .98,  .573,  .58}22 \\
		\hline
		BB30021 & \cellcolor[rgb]{ .988,  1,  .992}0.36 & \cellcolor[rgb]{ .384,  .745,  .478}0.32 & \cellcolor[rgb]{ .663,  .863,  .714}0.34 & \cellcolor[rgb]{ .98,  .69,  .702}16 \\
		\hline
		BB30026 & \cellcolor[rgb]{ .988,  1,  .992}0.19 & \cellcolor[rgb]{ .384,  .745,  .478}0.15 & \cellcolor[rgb]{ .741,  .894,  .78}0.18 & \cellcolor[rgb]{ .984,  .831,  .843}9 \\
		\hline
		BB30011 & \cellcolor[rgb]{ .988,  1,  .992}0.35 & \cellcolor[rgb]{ .384,  .745,  .478}0.31 & \cellcolor[rgb]{ .894,  .957,  .91}0.35 & \cellcolor[rgb]{ .973,  .412,  .42}30 \\
		\hline
		BB40009 & \cellcolor[rgb]{ .988,  1,  .992}0.19 & \cellcolor[rgb]{ .384,  .745,  .478}0.13 & \cellcolor[rgb]{ .945,  .98,  .957}0.18 & \cellcolor[rgb]{ .984,  .714,  .722}15 \\
		\hline
		BB40019 & \cellcolor[rgb]{ .988,  1,  .992}0.43 & \cellcolor[rgb]{ .384,  .745,  .478}0.29 & \cellcolor[rgb]{ .725,  .886,  .769}0.37 & \cellcolor[rgb]{ .973,  .412,  .42}30 \\
		\hline
		BB40033 & \cellcolor[rgb]{ .988,  1,  .992}0.13 & \cellcolor[rgb]{ .384,  .745,  .478}0.06 & \cellcolor[rgb]{ .706,  .878,  .753}0.10 & \cellcolor[rgb]{ .976,  .455,  .463}28 \\
		\hline
		BB40006 & \cellcolor[rgb]{ .988,  1,  .992}0.00 & \cellcolor[rgb]{ .988,  1,  .992}0.00 & \cellcolor[rgb]{ .988,  1,  .992}0.00 & \cellcolor[rgb]{ .988,  .871,  .882}7 \\
		\hline
		BB50002 & \cellcolor[rgb]{ .988,  1,  .992}0.50 & \cellcolor[rgb]{ .384,  .745,  .478}0.40 & \cellcolor[rgb]{ .941,  .98,  .953}0.49 & \cellcolor[rgb]{ .984,  .733,  .741}14 \\
		\hline
		BB50009 & \cellcolor[rgb]{ .988,  1,  .992}0.24 & \cellcolor[rgb]{ .384,  .745,  .478}0.08 & \cellcolor[rgb]{ .788,  .914,  .82}0.19 & \cellcolor[rgb]{ .98,  .631,  .643}19 \\
		\hline
		BB50014 & \cellcolor[rgb]{ .988,  1,  .992}0.41 & \cellcolor[rgb]{ .384,  .745,  .478}0.26 & \cellcolor[rgb]{ .835,  .933,  .863}0.37 & \cellcolor[rgb]{ .984,  .812,  .824}10 \\
		\hline
		BB50006 & \cellcolor[rgb]{ .988,  1,  .992}0.21 & \cellcolor[rgb]{ .384,  .745,  .478}0.12 & \cellcolor[rgb]{ .725,  .886,  .769}0.17 & \cellcolor[rgb]{ .988,  .89,  .902}6 \\
		\hline
	\end{tabular}%
	\label{tab:pmao-pasta-b}%
\end{table}%

\subsubsection{ML score vs. FN rate}
To illustrate a particular advantage of PMAO over PASTA, we plot the (ML score, FN rate) values of PASTA output and 30 solutions in the tree-space generated by PMAO in Figure~\ref{fig:ml-fn} for 10 arbitrary datasets. Such plot for all datasets of set A and set B are available in Supplementary Figures~\ref{fig:ml-fn-a}, \ref{fig:ml-fn-b}. There we find several cases where a PMAO solution having better ML score than PASTA but with a worse FN rate.  On the other hand, the best FN rate might correspond to a ML score much lower than PASTA. Similar observations can be shown for other objectives as well. These results illustrate the phenomena that entirely depending on a single optimization criterion (e.g., ML score by PASTA) may severely lead affect the accuracy. PMAO reduces such risk by employing many objectives.

\begin{figure*}[!htbp]%
	\begin{adjustwidth}{-1cm}{}
		\centering
		\subfloat[BB11018]{\includegraphics[width=0.22\textwidth]{PMAO-A/BB11018-fn-ml}}
		\subfloat[BB12001]{\includegraphics[width=0.22\textwidth]{PMAO-A/BB12001-fn-ml}}
		\subfloat[BB20010]{\includegraphics[width=0.22\textwidth]{PMAO-A/BB20010-fn-ml}}%
		\subfloat[BB20041]{\includegraphics[width=0.22\textwidth]{PMAO-A/BB20041-fn-ml}}%
		\subfloat[BB30002]{\includegraphics[width=0.22\textwidth]{PMAO-A/BB30002-fn-ml}}\\%
		\subfloat[BB30008]{\includegraphics[width=0.22\textwidth]{PMAO-A/BB30008-fn-ml}}%
		\subfloat[BB40001]{\includegraphics[width=0.22\textwidth]{PMAO-A/BB40001-fn-ml}}%
		\subfloat[BB40048]{\includegraphics[width=0.22\textwidth]{PMAO-A/BB40048-fn-ml}}%
		\subfloat[BB50005]{\includegraphics[width=0.22\textwidth]{PMAO-A/BB50005-fn-ml}}%	
		\subfloat[BB50016]{\includegraphics[width=0.22\textwidth]{PMAO-A/BB50016-fn-ml}}%
	\end{adjustwidth}
	\caption{Visualization of PASTA output and the 30 solutions generated by PMAO on 10 arbitrary datasets. The x-axis and y-axis represent ML score and FN rate respectively.}
	\label{fig:ml-fn}
\end{figure*}

\subsubsection{Weight vectors leading to better accuracy}
Now we the examine the flexibility of decomposition-based MO strategy within PMAO to address diverse characteristics varying across datasets. So we depict those weight vectors that resulted better or equivalent solutions to PASTA for four arbitrary datasets in Figures~\ref{fig:good-weight} where each line represents a weight vector along with its achieved FN rate. Similar plots for all datasets of set A and set B are available in Supplementary Figures~\ref{fig:good-weight-a}, \ref{fig:good-weight-b}. Note that the weight vectors yielding best FN rates greatly differs across the datasets and nearly all weight vectors consist of two or more non-zero values. These justifies our motivation for employing the MO approach and shows the robustness of PMAO to varying traits of the datasets. 

\begin{figure}[!htbp]%
	\begin{adjustwidth}{-1cm}{}
		\centering
		%\subfloat[BB11018]{\includegraphics[width=0.22\textwidth]{weight-A/BB11018-good-weight}}
		\subfloat[BB12001]{\includegraphics[width=0.25\textwidth]{weight-A/BB12001-good-weight}}
		%\subfloat[BB20010]{\includegraphics[width=0.22\textwidth]{weight-A/BB20010-good-weight}}%
		\subfloat[BB20041]{\includegraphics[width=0.25\textwidth]{weight-A/BB20041-good-weight}}\\
		%\subfloat[BB30002]{\includegraphics[width=0.22\textwidth]{weight-A/BB30002-good-weight}}\\%
		\subfloat[BB30008]{\includegraphics[width=0.25\textwidth]{weight-A/BB30008-good-weight}}%
		\subfloat[BB40001]{\includegraphics[width=0.25\textwidth]{weight-A/BB40001-good-weight}}%
		%\subfloat[BB40048]{\includegraphics[width=0.22\textwidth]{weight-A/BB40048-good-weight}}%
		%\subfloat[BB50005]{\includegraphics[width=0.22\textwidth]{weight-A/BB50005-good-weight}}%	
		%\subfloat[BB50016]{\label{3x}\includegraphics[width=0.22\textwidth]{weight-A/BB50016-good-weight}}%
	\end{adjustwidth}
	\caption{Visualization of weight vectors which lead PMAO to generate better or equivalent solutions to PASTA on four arbitrary datasets. The y-axis portrays the weight values and the x-axis marks the achieved FN rate. The weigh vectors are sorted in ascending order based on the achieved FN rate. }
	\label{fig:good-weight}
\end{figure}


\subsection{Machine learning based detection of few wight vectors}
%Examining 30 %Now we an
Here we address the question that, whether it is possible to detect a few weight vectors for PMAO, based on the features of the input unaligned sequences, such that the resultant small tree-space contains at least one high quality tree. It allows the domain expert to single out the best solution with manageable effort. To pursue this goal, we design a supervised regression based pipeline to detect five potential weight vectors comprising the following steps.
\begin{itemize}
	\item Step 1: Prepare training data to learn a supervised regression model of the form, $FN\_rate = f(X)$, based on our obtained results from each dataset under set A. Here the 15D feature vector $X$ consists of five weight values plus 10 features (used in~\cite{rubio2018characteristic}) extracted from the unaligned sequences. More details can be found in Appendix~\ref{appendex:train}. 
	\item Step 2: Fit an XGBoost regressor~\cite{bibid}, a popular gradient boosting method, on the training data prepared in Step 1. For details please see Appendix~\ref{appendex:train}. 
	\item Step 3: Calculate 100 well-spaced 5D weight vectors using the method of~\cite{ref_dirs_energy}.
	\item Step 4: Prepare a test data for each dataset under set B by appending 10 extracted features to each of the 100 weight vectors.
	\item Step 5: Predict the FN rate for each test feature vector formed in Step 4 using the model learned in Step 2. 
	\item Step 6: For each dataset in set B, select the five test feature vector with the least predicted FN rates and output the corresponding weight vectors.
\end{itemize}

The five weight vectors detected for each dataset under set B using the aforementioned approach are depicted in Supplementary Figure~\ref{label}. We also randomly picked five weight vectors for each dataset from the 30 weight vectors of Figure~\ref{fig:weight}. Thus we created two variants of PMAO listed in Table~\ref{tab:variants-weight}. We show the best and average FN rate achieved by these two variants along with PASTA's FN rate on datasets under set B Table~\ref{tab:pmao-5pw-b}. (Some more text). Significant difference between the best of PMAO-5DW and PASTA shows that potential this approach. 

\begin{table}[!htbp]
	\small
	\caption{PMAO variants based on selection of five input weight vectors.}
	\begin{tabular}{l|L{6cm}}
		Variant &  Weight vector selection scheme\\
		\hline
		5DW  &  Detected from 100 well-spaced vectors based on a machine learning model \\
		\hline
		5RW  &  Selected randomly from 30 well-spaced vectors \\
	\end{tabular}%
	\label{tab:variants-weight}%
\end{table}%



% Table generated by Excel2LaTeX from sheet 'stat-8I-30w-ML-w'
\begin{table}[!htbp]
	\small
	\caption{Comparison of the five solutions generated by two PMAO variants with respect to PASTA in terms of FN rate on set A datasets. For PMAO, we show the best and average FN rate 
		of the five solutions. The better values are marked with darker shade.}
	\begin{tabular}{|l|r|r|r|r|r|}
		\hline
		\multirow{2}{*}{Dataset} & \multirow{2}{*}{\makecell{PASTA\\FN rate}} & \multicolumn{2}{c|}{\makecell{PMO-5DW\\FN rate}} & \multicolumn{2}{c|}{\makecell{PMO-5RW\\FN rate}} \\
		\cline{3-6}          &       & Best & Avg & Best & Avg \\
		\hline
		BB11007 & \cellcolor[rgb]{ .686,  .871,  .733}0.50 & \cellcolor[rgb]{ .686,  .871,  .733}0.50 & \cellcolor[rgb]{ .988,  1,  .992}0.67 & \cellcolor[rgb]{ .384,  .745,  .478}0.33 & \cellcolor[rgb]{ .804,  .922,  .835}0.57 \\
		\hline
		BB11034 & \cellcolor[rgb]{ .635,  .851,  .69}0.40 & \cellcolor[rgb]{ .384,  .745,  .478}0.20 & \cellcolor[rgb]{ .584,  .827,  .647}0.36 & \cellcolor[rgb]{ .384,  .745,  .478}0.20 & \cellcolor[rgb]{ .988,  1,  .992}0.68 \\
		\hline
		BB11038 & \cellcolor[rgb]{ .384,  .745,  .478}0.40 & \cellcolor[rgb]{ .384,  .745,  .478}0.40 & \cellcolor[rgb]{ .988,  1,  .992}0.64 & \cellcolor[rgb]{ .384,  .745,  .478}0.40 & \cellcolor[rgb]{ .686,  .871,  .733}0.52 \\
		\hline
		BB11019 & \cellcolor[rgb]{ .384,  .745,  .478}0.29 & \cellcolor[rgb]{ .384,  .745,  .478}0.29 & \cellcolor[rgb]{ .384,  .745,  .478}0.29 & \cellcolor[rgb]{ .384,  .745,  .478}0.29 & \cellcolor[rgb]{ .988,  1,  .992}0.31 \\
		\hline
		BB12005 & \cellcolor[rgb]{ .384,  .745,  .478}0.33 & \cellcolor[rgb]{ .384,  .745,  .478}0.33 & \cellcolor[rgb]{ .988,  1,  .992}0.37 & \cellcolor[rgb]{ .384,  .745,  .478}0.33 & \cellcolor[rgb]{ .988,  1,  .992}0.37 \\
		\hline
		BB12029 & \cellcolor[rgb]{ .988,  1,  .992}0.44 & \cellcolor[rgb]{ .384,  .745,  .478}0.33 & \cellcolor[rgb]{ .745,  .894,  .784}0.40 & \cellcolor[rgb]{ .988,  1,  .992}0.44 & \cellcolor[rgb]{ .988,  1,  .992}0.44 \\
		\hline
		BB12026 & \cellcolor[rgb]{ .682,  .871,  .733}0.53 & \cellcolor[rgb]{ .384,  .745,  .478}0.47 & \cellcolor[rgb]{ .682,  .871,  .733}0.53 & \cellcolor[rgb]{ .682,  .871,  .733}0.53 & \cellcolor[rgb]{ .988,  1,  .992}0.60 \\
		\hline
		BB12037 & \cellcolor[rgb]{ .384,  .745,  .478}0.40 & \cellcolor[rgb]{ .914,  .969,  .929}0.70 & \cellcolor[rgb]{ .988,  1,  .992}0.74 & \cellcolor[rgb]{ .384,  .745,  .478}0.40 & \cellcolor[rgb]{ .737,  .894,  .78}0.60 \\
		\hline
		BB20002 & \cellcolor[rgb]{ .549,  .816,  .62}0.65 & \cellcolor[rgb]{ .384,  .745,  .478}0.59 & \cellcolor[rgb]{ .651,  .855,  .706}0.68 & \cellcolor[rgb]{ .384,  .745,  .478}0.59 & \cellcolor[rgb]{ .988,  1,  .992}0.80 \\
		\hline
		BB20012 & \cellcolor[rgb]{ .988,  1,  .992}0.29 & \cellcolor[rgb]{ .384,  .745,  .478}0.17 & \cellcolor[rgb]{ .784,  .914,  .82}0.25 & \cellcolor[rgb]{ .584,  .827,  .647}0.21 & \cellcolor[rgb]{ .824,  .929,  .855}0.26 \\
		\hline
		BB20030 & \cellcolor[rgb]{ .384,  .745,  .478}0.64 & \cellcolor[rgb]{ .384,  .745,  .478}0.64 & \cellcolor[rgb]{ .988,  1,  .992}0.69 & \cellcolor[rgb]{ .384,  .745,  .478}0.64 & \cellcolor[rgb]{ .933,  .976,  .945}0.68 \\
		\hline
		BB20037 & \cellcolor[rgb]{ .384,  .745,  .478}0.13 & \cellcolor[rgb]{ .549,  .816,  .62}0.16 & \cellcolor[rgb]{ .8,  .922,  .831}0.21 & \cellcolor[rgb]{ .886,  .957,  .906}0.23 & \cellcolor[rgb]{ .988,  1,  .992}0.25 \\
		\hline
		BB30003 & \cellcolor[rgb]{ .988,  1,  .992}0.30 & \cellcolor[rgb]{ .384,  .745,  .478}0.28 & \cellcolor[rgb]{ .706,  .878,  .749}0.29 & \cellcolor[rgb]{ .584,  .827,  .647}0.29 & \cellcolor[rgb]{ .945,  .98,  .957}0.30 \\
		\hline
		BB30021 & \cellcolor[rgb]{ .941,  .98,  .953}0.36 & \cellcolor[rgb]{ .384,  .745,  .478}0.32 & \cellcolor[rgb]{ .988,  1,  .992}0.36 & \cellcolor[rgb]{ .384,  .745,  .478}0.32 & \cellcolor[rgb]{ .784,  .914,  .82}0.35 \\
		\hline
		BB30026 & \cellcolor[rgb]{ .859,  .945,  .882}0.19 & \cellcolor[rgb]{ .384,  .745,  .478}0.15 & \cellcolor[rgb]{ .922,  .973,  .937}0.20 & \cellcolor[rgb]{ .702,  .878,  .745}0.18 & \cellcolor[rgb]{ .988,  1,  .992}0.20 \\
		\hline
		BB30011 & \cellcolor[rgb]{ .988,  1,  .992}0.35 & \cellcolor[rgb]{ .384,  .745,  .478}0.32 & \cellcolor[rgb]{ .804,  .922,  .835}0.35 & \cellcolor[rgb]{ .988,  1,  .992}0.35 & \cellcolor[rgb]{ .988,  1,  .992}0.35 \\
		\hline
		BB40009 & \cellcolor[rgb]{ .384,  .745,  .478}0.19 & \cellcolor[rgb]{ .384,  .745,  .478}0.19 & \cellcolor[rgb]{ .533,  .808,  .604}0.20 & \cellcolor[rgb]{ .384,  .745,  .478}0.19 & \cellcolor[rgb]{ .988,  1,  .992}0.24 \\
		\hline
		BB40019 & \cellcolor[rgb]{ .988,  1,  .992}0.43 & \cellcolor[rgb]{ .384,  .745,  .478}0.29 & \cellcolor[rgb]{ .745,  .898,  .784}0.37 & \cellcolor[rgb]{ .384,  .745,  .478}0.29 & \cellcolor[rgb]{ .624,  .847,  .682}0.34 \\
		\hline
		BB40033 & \cellcolor[rgb]{ .988,  1,  .992}0.13 & \cellcolor[rgb]{ .384,  .745,  .478}0.06 & \cellcolor[rgb]{ .867,  .949,  .886}0.11 & \cellcolor[rgb]{ .384,  .745,  .478}0.06 & \cellcolor[rgb]{ .867,  .949,  .886}0.11 \\
		\hline
		BB40006 & \cellcolor[rgb]{ .384,  .745,  .478}0.00 & \cellcolor[rgb]{ .384,  .745,  .478}0.00 & \cellcolor[rgb]{ .886,  .957,  .906}0.09 & \cellcolor[rgb]{ .886,  .957,  .906}0.09 & \cellcolor[rgb]{ .988,  1,  .992}0.11 \\
		\hline
		BB50002 & \cellcolor[rgb]{ .718,  .886,  .761}0.50 & \cellcolor[rgb]{ .384,  .745,  .478}0.40 & \cellcolor[rgb]{ .851,  .941,  .875}0.54 & \cellcolor[rgb]{ .718,  .886,  .761}0.50 & \cellcolor[rgb]{ .988,  1,  .992}0.58 \\
		\hline
		BB50009 & \cellcolor[rgb]{ .988,  1,  .992}0.24 & \cellcolor[rgb]{ .835,  .933,  .863}0.20 & \cellcolor[rgb]{ .957,  .984,  .965}0.23 & \cellcolor[rgb]{ .384,  .745,  .478}0.08 & \cellcolor[rgb]{ .835,  .933,  .863}0.20 \\
		\hline
		BB50014 & \cellcolor[rgb]{ .384,  .745,  .478}0.41 & \cellcolor[rgb]{ .384,  .745,  .478}0.41 & \cellcolor[rgb]{ .867,  .949,  .886}0.44 & \cellcolor[rgb]{ .988,  1,  .992}0.44 & \cellcolor[rgb]{ .988,  1,  .992}0.44 \\
		\hline
		BB50006 & \cellcolor[rgb]{ .749,  .898,  .788}0.21 & \cellcolor[rgb]{ .384,  .745,  .478}0.14 & \cellcolor[rgb]{ .839,  .937,  .867}0.23 & \cellcolor[rgb]{ .655,  .859,  .71}0.19 & \cellcolor[rgb]{ .988,  1,  .992}0.26 \\
		\hline
	\end{tabular}%
	\label{tab:pmao-5pw-b}%
\end{table}%

% Table generated by Excel2LaTeX from sheet 'stat test'
\begin{table}[!htbp]
	\small
	\caption{\underline{Friedman Aligned Ranks test (Column 2):} Friedman Aligned ranks (lower is better) of PASTA and the two PMAO variants based on the best FN rates reported in Table~\ref{tab:pmao-5pw-b}. We also show the computed statistics and corresponding $ p $-value. 
	\underline{Holm's post-hoc procedure (Columns 3 - 5):} Comparison among PASTA and two PMAO variants using the Holm's post-hoc procedures. Each entry shows the adjusted $p$-value which indicates the significance of the difference in performance between two methods.}
	\begin{tabular}{|l|r||ccc|}
		\hline
		\multicolumn{1}{|c|}{1} & \multicolumn{1}{c||}{2} & \multicolumn{1}{c|}{3} & \multicolumn{1}{c|}{4} & 5 \\
		\hline
		\multirow{2}{*}{\makecell{Method}} & \multirow{2}{*}{\makecell{Friedman\\Aligned Rank*}} & \multicolumn{3}{c|}{Holm's adjusted $p$-value} \\
		\cline{3-5}          &       & \multicolumn{1}{l|}{5DW} & \multicolumn{1}{l|}{5RW} & \multicolumn{1}{l|}{PASTA} \\
		\hline
		5DW & 27.0000 & \multicolumn{1}{c|}{-} & \multicolumn{1}{r|}{0.1972} & \multicolumn{1}{r|}{\cellcolor[rgb]{ .384,  .745,  .478}0.0018} \\
		\hline
		5RW & 34.7917 & \multicolumn{1}{r|}{0.1972} & \multicolumn{1}{c|}{-} & \multicolumn{1}{r|}{0.0650} \\
		\hline
		PASTA & 47.7083 & \multicolumn{1}{r|}{\cellcolor[rgb]{ .384,  .745,  .478}0.0018} & \multicolumn{1}{r|}{0.0650} & - \\
		\hline
		*Statistic & 8.5117 & \multicolumn{3}{c|}{\multirow{2}{*}{N/A}} \\
		\cline{1-2}    *$p$-value & 0.0142 & \multicolumn{3}{c|}{} \\
		\hline
	\end{tabular}%
	\label{tab:test-pasta-variants}%
\end{table}%

\subsection{Obtain one solution from PMAO}
We employ three schemes to obtain a single solution from the 30 solutions generated by PMAO
\begin{enumerate}
	\item NSUM: Select the solution whose sum of normalized (min-max) objective values is the maximum among the 30 candidates
	\item GRED: Summarize the 30 candidate trees using greedy consensus of PAUP
	\item AST: Summarize the 30 candidate trees using ASTRAL, a popular tool for quartet based stigmatization method
\end{enumerate}

Statistical test: no sig diff between any pair Table~\ref{tab:select-one-a}, \ref{tab:select-one-b} but AST exhibits a higher rank others
% Table generated by Excel2LaTeX from sheet 'Sheet7 (2)'
\begin{table}[!htbp]
	\small
	\caption{Comparison of the three selection schemes used to obtain a single solution from the 30 solutions generated by PMAO on datasets under set A in terms of FN rate. We also include PASTA's FN rate and PMAO best FN rate. The better values are marked with darker shade.}
	\begin{tabular}{|l|r|r|r|r|r|}
		\hline
		\multirow{2}{*}{Dataset} & \multirow{2}{*}{PASTA} & \multicolumn{3}{c|}{Selection scheme} & \multirow{2}{*}{\makecell{PMAO\\Best}} \\
		\cline{3-5}          &       & \multicolumn{1}{l|}{NSUM} & \multicolumn{1}{l|}{GRED} & \multicolumn{1}{l|}{AST} &  \\
		\hline
		BB11005 & \cellcolor[rgb]{ .988,  1,  .992}0.55 & \cellcolor[rgb]{ .431,  .824,  .51}0.27 & \cellcolor[rgb]{ .431,  .824,  .51}0.27 & \cellcolor[rgb]{ .431,  .824,  .51}0.27 & \cellcolor[rgb]{ .384,  .725,  .478}0.09 \\
		\hline
		BB11018 & \cellcolor[rgb]{ .988,  1,  .992}0.27 & \cellcolor[rgb]{ .988,  1,  .992}0.27 & \cellcolor[rgb]{ .988,  1,  .992}0.27 & \cellcolor[rgb]{ .988,  1,  .992}0.27 & \cellcolor[rgb]{ .384,  .725,  .478}0.18 \\
		\hline
		BB11033 & \cellcolor[rgb]{ .384,  .725,  .478}0.38 & \cellcolor[rgb]{ .988,  1,  .992}0.50 & \cellcolor[rgb]{ .988,  1,  .992}0.50 & \cellcolor[rgb]{ .988,  1,  .992}0.50 & \cellcolor[rgb]{ .384,  .725,  .478}0.38 \\
		\hline
		BB11020 & \cellcolor[rgb]{ .988,  1,  .992}0.83 & \cellcolor[rgb]{ .431,  .824,  .51}0.67 & \cellcolor[rgb]{ .431,  .824,  .51}0.67 & \cellcolor[rgb]{ .431,  .824,  .51}0.67 & \cellcolor[rgb]{ .384,  .725,  .478}0.33 \\
		\hline
		BB12001 & \cellcolor[rgb]{ .431,  .824,  .51}0.25 & \cellcolor[rgb]{ .384,  .725,  .478}0.13 & \cellcolor[rgb]{ .988,  1,  .992}0.38 & \cellcolor[rgb]{ .988,  1,  .992}0.38 & \cellcolor[rgb]{ .384,  .725,  .478}0.13 \\
		\hline
		BB12013 & \cellcolor[rgb]{ .988,  1,  .992}0.20 & \cellcolor[rgb]{ .988,  1,  .992}0.20 & \cellcolor[rgb]{ .988,  1,  .992}0.20 & \cellcolor[rgb]{ .988,  1,  .992}0.20 & \cellcolor[rgb]{ .988,  1,  .992}0.20 \\
		\hline
		BB12022 & \cellcolor[rgb]{ .988,  1,  .992}0.00 & \cellcolor[rgb]{ .988,  1,  .992}0.00 & \cellcolor[rgb]{ .988,  1,  .992}0.00 & \cellcolor[rgb]{ .988,  1,  .992}0.00 & \cellcolor[rgb]{ .988,  1,  .992}0.00 \\
		\hline
		BB12035 & \cellcolor[rgb]{ .384,  .725,  .478}0.00 & \cellcolor[rgb]{ .988,  1,  .992}0.46 & \cellcolor[rgb]{ .431,  .824,  .51}0.04 & \cellcolor[rgb]{ .431,  .824,  .51}0.04 & \cellcolor[rgb]{ .431,  .824,  .51}0.04 \\
		\hline
		BB12044 & \cellcolor[rgb]{ .988,  1,  .992}0.50 & \cellcolor[rgb]{ .988,  1,  .992}0.50 & \cellcolor[rgb]{ .988,  1,  .992}0.50 & \cellcolor[rgb]{ .988,  1,  .992}0.50 & \cellcolor[rgb]{ .384,  .725,  .478}0.38 \\
		\hline
		BB20001 & \cellcolor[rgb]{ .988,  1,  .992}0.54 & \cellcolor[rgb]{ .431,  .824,  .51}0.46 & \cellcolor[rgb]{ .431,  .824,  .51}0.46 & \cellcolor[rgb]{ .431,  .824,  .51}0.46 & \cellcolor[rgb]{ .384,  .725,  .478}0.23 \\
		\hline
		BB20010 & \cellcolor[rgb]{ .431,  .824,  .51}0.35 & \cellcolor[rgb]{ .424,  .808,  .502}0.31 & \cellcolor[rgb]{ .988,  1,  .992}0.38 & \cellcolor[rgb]{ .431,  .824,  .51}0.35 & \cellcolor[rgb]{ .384,  .725,  .478}0.08 \\
		\hline
		BB20022 & \cellcolor[rgb]{ .431,  .824,  .51}0.11 & \cellcolor[rgb]{ .984,  .996,  .988}0.13 & \cellcolor[rgb]{ .988,  1,  .992}0.13 & \cellcolor[rgb]{ .431,  .824,  .51}0.11 & \cellcolor[rgb]{ .384,  .725,  .478}0.09 \\
		\hline
		BB20033 & \cellcolor[rgb]{ .431,  .824,  .51}0.36 & \cellcolor[rgb]{ .384,  .725,  .478}0.24 & \cellcolor[rgb]{ .988,  1,  .992}0.38 & \cellcolor[rgb]{ .988,  1,  .992}0.38 & \cellcolor[rgb]{ .384,  .725,  .478}0.24 \\
		\hline
		BB20041 & \cellcolor[rgb]{ .431,  .824,  .51}0.38 & \cellcolor[rgb]{ .988,  1,  .992}0.44 & \cellcolor[rgb]{ .431,  .824,  .51}0.38 & \cellcolor[rgb]{ .408,  .773,  .494}0.36 & \cellcolor[rgb]{ .384,  .725,  .478}0.33 \\
		\hline
		BB30002 & \cellcolor[rgb]{ .431,  .824,  .51}0.32 & \cellcolor[rgb]{ .988,  1,  .992}0.36 & \cellcolor[rgb]{ .988,  1,  .992}0.36 & \cellcolor[rgb]{ .431,  .824,  .51}0.32 & \cellcolor[rgb]{ .384,  .725,  .478}0.18 \\
		\hline
		BB30008 & \cellcolor[rgb]{ .427,  .82,  .506}0.33 & \cellcolor[rgb]{ .988,  1,  .992}0.45 & \cellcolor[rgb]{ .569,  .867,  .627}0.36 & \cellcolor[rgb]{ .431,  .824,  .51}0.33 & \cellcolor[rgb]{ .384,  .725,  .478}0.21 \\
		\hline
		BB30015 & \cellcolor[rgb]{ .988,  1,  .992}0.17 & \cellcolor[rgb]{ .988,  1,  .992}0.17 & \cellcolor[rgb]{ .988,  1,  .992}0.17 & \cellcolor[rgb]{ .988,  1,  .992}0.17 & \cellcolor[rgb]{ .988,  1,  .992}0.17 \\
		\hline
		BB30022 & \cellcolor[rgb]{ .988,  1,  .992}0.51 & \cellcolor[rgb]{ .412,  .788,  .498}0.49 & \cellcolor[rgb]{ .988,  1,  .992}0.51 & \cellcolor[rgb]{ .988,  1,  .992}0.51 & \cellcolor[rgb]{ .384,  .725,  .478}0.46 \\
		\hline
		BB40001 & \cellcolor[rgb]{ .431,  .824,  .51}0.48 & \cellcolor[rgb]{ .431,  .824,  .51}0.48 & \cellcolor[rgb]{ .988,  1,  .992}0.52 & \cellcolor[rgb]{ .988,  1,  .992}0.52 & \cellcolor[rgb]{ .384,  .725,  .478}0.36 \\
		\hline
		BB40013 & \cellcolor[rgb]{ .616,  .878,  .667}0.38 & \cellcolor[rgb]{ .988,  1,  .992}0.44 & \cellcolor[rgb]{ .431,  .824,  .51}0.34 & \cellcolor[rgb]{ .416,  .788,  .498}0.31 & \cellcolor[rgb]{ .384,  .725,  .478}0.25 \\
		\hline
		BB40025 & \cellcolor[rgb]{ .988,  1,  .992}0.00 & \cellcolor[rgb]{ .988,  1,  .992}0.00 & \cellcolor[rgb]{ .988,  1,  .992}0.00 & \cellcolor[rgb]{ .988,  1,  .992}0.00 & \cellcolor[rgb]{ .988,  1,  .992}0.00 \\
		\hline
		BB40038 & \cellcolor[rgb]{ .988,  1,  .992}0.25 & \cellcolor[rgb]{ .416,  .788,  .498}0.20 & \cellcolor[rgb]{ .988,  1,  .992}0.25 & \cellcolor[rgb]{ .988,  1,  .992}0.25 & \cellcolor[rgb]{ .384,  .725,  .478}0.10 \\
		\hline
		BB40048 & \cellcolor[rgb]{ .988,  1,  .992}0.43 & \cellcolor[rgb]{ .384,  .725,  .478}0.29 & \cellcolor[rgb]{ .384,  .725,  .478}0.29 & \cellcolor[rgb]{ .384,  .725,  .478}0.29 & \cellcolor[rgb]{ .384,  .725,  .478}0.29 \\
		\hline
		BB50001 & \cellcolor[rgb]{ .384,  .725,  .478}0.29 & \cellcolor[rgb]{ .988,  1,  .992}0.32 & \cellcolor[rgb]{ .384,  .725,  .478}0.29 & \cellcolor[rgb]{ .384,  .725,  .478}0.29 & \cellcolor[rgb]{ .384,  .725,  .478}0.29 \\
		\hline
		BB50005 & \cellcolor[rgb]{ .988,  1,  .992}0.38 & \cellcolor[rgb]{ .988,  1,  .992}0.38 & \cellcolor[rgb]{ .988,  1,  .992}0.38 & \cellcolor[rgb]{ .988,  1,  .992}0.38 & \cellcolor[rgb]{ .384,  .725,  .478}0.25 \\
		\hline
		BB50010 & \cellcolor[rgb]{ .384,  .725,  .478}0.00 & \cellcolor[rgb]{ .988,  1,  .992}0.14 & \cellcolor[rgb]{ .988,  1,  .992}0.14 & \cellcolor[rgb]{ .988,  1,  .992}0.14 & \cellcolor[rgb]{ .384,  .725,  .478}0.00 \\
		\hline
		BB50016 & \cellcolor[rgb]{ .988,  1,  .992}0.47 & \cellcolor[rgb]{ .431,  .824,  .51}0.13 & \cellcolor[rgb]{ .431,  .824,  .51}0.13 & \cellcolor[rgb]{ .431,  .824,  .51}0.13 & \cellcolor[rgb]{ .384,  .725,  .478}0.07 \\
		\hline
	\end{tabular}%
	\label{tab:select-one-a}%
\end{table}%


% Table generated by Excel2LaTeX from sheet 'Sheet7 (2)'
\begin{table}[!htbp]
	\small
	\caption{Comparison of the three selection schemes used to obtain a single solution from the 30 solutions generated by PMAO on datasets under set A in terms of FN rate. We also include PASTA's FN rate and PMAO best FN rate. The better values are marked with darker shade.}
	\begin{tabular}{|l|r|r|r|r|r|}
		\hline
		\multirow{2}{*}{Dataset} & \multirow{2}{*}{PASTA} & \multicolumn{3}{c|}{Selection scheme} & \multirow{2}{*}{\makecell{PMAO\\Best}} \\
		\cline{3-5}          &       & \multicolumn{1}{l|}{NSUM} & \multicolumn{1}{l|}{GRED} & \multicolumn{1}{l|}{AST} &  \\
		\hline
		BB11007 & \cellcolor[rgb]{ .408,  .773,  .494}0.50 & \cellcolor[rgb]{ .988,  1,  .992}0.83 & \cellcolor[rgb]{ .431,  .824,  .51}0.67 & \cellcolor[rgb]{ .431,  .824,  .51}0.67 & \cellcolor[rgb]{ .384,  .725,  .478}0.33 \\
		\hline
		BB11034 & \cellcolor[rgb]{ .431,  .824,  .51}0.40 & \cellcolor[rgb]{ .988,  1,  .992}1.00 & \cellcolor[rgb]{ .431,  .824,  .51}0.40 & \cellcolor[rgb]{ .431,  .824,  .51}0.40 & \cellcolor[rgb]{ .384,  .725,  .478}0.20 \\
		\hline
		BB11038 & \cellcolor[rgb]{ .988,  1,  .992}0.40 & \cellcolor[rgb]{ .988,  1,  .992}0.40 & \cellcolor[rgb]{ .988,  1,  .992}0.40 & \cellcolor[rgb]{ .988,  1,  .992}0.40 & \cellcolor[rgb]{ .384,  .725,  .478}0.00 \\
		\hline
		BB11019 & \cellcolor[rgb]{ .988,  1,  .992}0.29 & \cellcolor[rgb]{ .988,  1,  .992}0.29 & \cellcolor[rgb]{ .988,  1,  .992}0.29 & \cellcolor[rgb]{ .988,  1,  .992}0.29 & \cellcolor[rgb]{ .384,  .725,  .478}0.14 \\
		\hline
		BB12005 & \cellcolor[rgb]{ .384,  .725,  .478}0.33 & \cellcolor[rgb]{ .384,  .725,  .478}0.33 & \cellcolor[rgb]{ .988,  1,  .992}0.33 & \cellcolor[rgb]{ .988,  1,  .992}0.33 & \cellcolor[rgb]{ .384,  .725,  .478}0.33 \\
		\hline
		BB12029 & \cellcolor[rgb]{ .988,  1,  .992}0.44 & \cellcolor[rgb]{ .988,  1,  .992}0.44 & \cellcolor[rgb]{ .988,  1,  .992}0.44 & \cellcolor[rgb]{ .988,  1,  .992}0.44 & \cellcolor[rgb]{ .384,  .725,  .478}0.33 \\
		\hline
		BB12026 & \cellcolor[rgb]{ .706,  .91,  .749}0.53 & \cellcolor[rgb]{ .988,  1,  .992}0.60 & \cellcolor[rgb]{ .431,  .824,  .51}0.47 & \cellcolor[rgb]{ .431,  .824,  .51}0.47 & \cellcolor[rgb]{ .384,  .725,  .478}0.33 \\
		\hline
		BB12037 & \cellcolor[rgb]{ .431,  .824,  .51}0.40 & \cellcolor[rgb]{ .431,  .824,  .51}0.40 & \cellcolor[rgb]{ .988,  1,  .992}0.60 & \cellcolor[rgb]{ .431,  .824,  .51}0.40 & \cellcolor[rgb]{ .384,  .725,  .478}0.10 \\
		\hline
		BB20002 & \cellcolor[rgb]{ .431,  .824,  .51}0.65 & \cellcolor[rgb]{ .988,  1,  .992}0.76 & \cellcolor[rgb]{ .431,  .824,  .51}0.65 & \cellcolor[rgb]{ .431,  .824,  .51}0.65 & \cellcolor[rgb]{ .384,  .725,  .478}0.53 \\
		\hline
		BB20012 & \cellcolor[rgb]{ .8,  .941,  .831}0.29 & \cellcolor[rgb]{ .988,  1,  .992}0.33 & \cellcolor[rgb]{ .431,  .824,  .51}0.21 & \cellcolor[rgb]{ .431,  .824,  .51}0.21 & \cellcolor[rgb]{ .384,  .725,  .478}0.08 \\
		\hline
		BB20030 & \cellcolor[rgb]{ .384,  .725,  .478}0.64 & \cellcolor[rgb]{ .384,  .725,  .478}0.64 & \cellcolor[rgb]{ .988,  1,  .992}0.70 & \cellcolor[rgb]{ .988,  1,  .992}0.70 & \cellcolor[rgb]{ .384,  .725,  .478}0.64 \\
		\hline
		BB20037 & \cellcolor[rgb]{ .384,  .725,  .478}0.13 & \cellcolor[rgb]{ .988,  1,  .992}0.27 & \cellcolor[rgb]{ .431,  .824,  .51}0.23 & \cellcolor[rgb]{ .431,  .824,  .51}0.23 & \cellcolor[rgb]{ .384,  .725,  .478}0.13 \\
		\hline
		BB30003 & \cellcolor[rgb]{ .8,  .937,  .827}0.30 & \cellcolor[rgb]{ .988,  1,  .992}0.31 & \cellcolor[rgb]{ .431,  .824,  .51}0.29 & \cellcolor[rgb]{ .431,  .824,  .51}0.29 & \cellcolor[rgb]{ .384,  .725,  .478}0.26 \\
		\hline
		BB30021 & \cellcolor[rgb]{ .416,  .792,  .498}0.36 & \cellcolor[rgb]{ .988,  1,  .992}0.39 & \cellcolor[rgb]{ .431,  .824,  .51}0.37 & \cellcolor[rgb]{ .431,  .824,  .51}0.37 & \cellcolor[rgb]{ .384,  .725,  .478}0.32 \\
		\hline
		BB30026 & \cellcolor[rgb]{ .42,  .796,  .502}0.19 & \cellcolor[rgb]{ .431,  .824,  .51}0.21 & \cellcolor[rgb]{ .988,  1,  .992}0.22 & \cellcolor[rgb]{ .988,  1,  .992}0.22 & \cellcolor[rgb]{ .384,  .725,  .478}0.15 \\
		\hline
		BB30011 & \cellcolor[rgb]{ .988,  1,  .992}0.35 & \cellcolor[rgb]{ .988,  1,  .992}0.35 & \cellcolor[rgb]{ .431,  .824,  .51}0.34 & \cellcolor[rgb]{ .431,  .824,  .51}0.34 & \cellcolor[rgb]{ .384,  .725,  .478}0.31 \\
		\hline
		BB40009 & \cellcolor[rgb]{ .988,  1,  .992}0.19 & \cellcolor[rgb]{ .988,  1,  .992}0.19 & \cellcolor[rgb]{ .384,  .725,  .478}0.13 & \cellcolor[rgb]{ .384,  .725,  .478}0.13 & \cellcolor[rgb]{ .384,  .725,  .478}0.13 \\
		\hline
		BB40019 & \cellcolor[rgb]{ .988,  1,  .992}0.43 & \cellcolor[rgb]{ .384,  .725,  .478}0.29 & \cellcolor[rgb]{ .988,  1,  .992}0.43 & \cellcolor[rgb]{ .988,  1,  .992}0.43 & \cellcolor[rgb]{ .384,  .725,  .478}0.29 \\
		\hline
		BB40033 & \cellcolor[rgb]{ .988,  1,  .992}0.13 & \cellcolor[rgb]{ .988,  1,  .992}0.13 & \cellcolor[rgb]{ .988,  1,  .992}0.13 & \cellcolor[rgb]{ .384,  .725,  .478}0.06 & \cellcolor[rgb]{ .384,  .725,  .478}0.06 \\
		\hline
		BB40006 & \cellcolor[rgb]{ .384,  .725,  .478}0.00 & \cellcolor[rgb]{ .384,  .725,  .478}0.00 & \cellcolor[rgb]{ .384,  .725,  .478}0.00 & \cellcolor[rgb]{ .988,  1,  .992}0.09 & \cellcolor[rgb]{ .384,  .725,  .478}0.00 \\
		\hline
		BB50002 & \cellcolor[rgb]{ .988,  1,  .992}0.50 & \cellcolor[rgb]{ .988,  1,  .992}0.50 & \cellcolor[rgb]{ .988,  1,  .992}0.50 & \cellcolor[rgb]{ .988,  1,  .992}0.50 & \cellcolor[rgb]{ .384,  .725,  .478}0.40 \\
		\hline
		BB50009 & \cellcolor[rgb]{ .616,  .878,  .667}0.24 & \cellcolor[rgb]{ .988,  1,  .992}0.32 & \cellcolor[rgb]{ .396,  .757,  .486}0.12 & \cellcolor[rgb]{ .431,  .824,  .51}0.20 & \cellcolor[rgb]{ .384,  .725,  .478}0.08 \\
		\hline
		BB50014 & \cellcolor[rgb]{ .42,  .804,  .502}0.41 & \cellcolor[rgb]{ .988,  1,  .992}0.44 & \cellcolor[rgb]{ .988,  1,  .992}0.44 & \cellcolor[rgb]{ .988,  1,  .992}0.44 & \cellcolor[rgb]{ .384,  .725,  .478}0.26 \\
		\hline
		BB50006 & \cellcolor[rgb]{ .431,  .824,  .51}0.21 & \cellcolor[rgb]{ .4,  .765,  .49}0.16 & \cellcolor[rgb]{ .988,  1,  .992}0.25 & \cellcolor[rgb]{ .988,  1,  .992}0.25 & \cellcolor[rgb]{ .384,  .725,  .478}0.12 \\
		\hline
	\end{tabular}%
	\label{tab:select-one-b}%
\end{table}%


\begin{comment}

\begin{figure*}%
\begin{adjustwidth}{-1cm}{}
\centering
\subfloat[Set A]{\includegraphics[width=1.1\textwidth]{partial-plot-A}}\\
\subfloat[Set B]{\includegraphics[width=1.1\textwidth]{partial-plot-B}}	
\end{adjustwidth}
\caption{Three sub-floats.}
\label{3figs}
\end{figure*}

\begin{figure*}[!htbp]
	\begin{adjustwidth}{-3cm}{-3cm}
		\centering
		
		\begin{subfigure}[b]{0.1\textwidth}
			\includegraphics[width=\columnwidth]{Figure/PMAO-A/BB11018-fn-ml}
			%\caption{BB11005}
			%\label{fig:con_pr09}
		\end{subfigure}    
		\begin{subfigure}[b]{0.25\textwidth}
			\includegraphics[width=\columnwidth]{Figure/PMAO-A/BB11018-fn-ml}
			%\caption{BB11018}
			%\label{fig:con_pr09}
		\end{subfigure}
		\begin{subfigure}[b]{0.25\textwidth}
			\includegraphics[width=\columnwidth]{Figure/PMAO-A/BB11018-fn-ml}
			%\caption{BB11020}
			%\label{fig:con_pr09}
		\end{subfigure}
		\begin{subfigure}[b]{0.25\textwidth}
			\includegraphics[width=\columnwidth]{Figure/PMAO-A/BB11018-fn-ml}
			%\caption{BB11033}
			%\label{fig:con_pr09}
		\end{subfigure}
		\begin{subfigure}[b]{0.25\textwidth}
			\includegraphics[width=\columnwidth]{Figure/PMAO-A/BB11018-fn-ml}
			%\caption{BB11033}
			%\label{fig:con_pr09}
		\end{subfigure}
		
		\begin{subfigure}[b]{0.25\textwidth}
			\includegraphics[width=\columnwidth]{Figure/PMAO-A/BB11018-fn-ml}
			%\caption{BB11005}
			%\label{fig:con_pr09}
		\end{subfigure}    
		\begin{subfigure}[b]{0.25\textwidth}
			\includegraphics[width=\columnwidth]{Figure/PMAO-A/BB11018-fn-ml}
			%\caption{BB11018}
			%\label{fig:con_pr09}
		\end{subfigure}
		\begin{subfigure}[b]{0.25\textwidth}
			\includegraphics[width=\columnwidth]{Figure/PMAO-A/BB11018-fn-ml}
			%\caption{BB11020}
			%\label{fig:con_pr09}
		\end{subfigure}
		\begin{subfigure}[b]{0.25\textwidth}
			\includegraphics[width=\columnwidth]{Figure/PMAO-A/BB11018-fn-ml}
			%\caption{BB11033}
			%\label{fig:con_pr09}
		\end{subfigure}
		\begin{subfigure}[b]{0.25\textwidth}
			\includegraphics[width=\columnwidth]{Figure/PMAO-A/BB11018-fn-ml}
			%\caption{BB11033}
			%\label{fig:con_pr09}
		\end{subfigure}
	
		\caption{ 3 iterations}
		\label{fig:fn_rate_3iter}
	\end{adjustwidth}
\end{figure*}
\end{comment}

