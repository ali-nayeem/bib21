\abstract{Multiple sequence alignment (MSA) is a prerequisite for several analyses in bioinformatics such as phylogeny estimation, protein structure prediction, etc. PASTA (Practical Alignments using SAT\'e and TrAnsitivity) is a state-of-the-art method for computing MSAs, well-known for its accuracy and scalability. It iteratively co-estimates both MSA and maximum likelihood (ML) phylogenetic tree. It attempts to exploit the close association between the accuracy of an MSA and the corresponding tree in finding the output through multiple iterations from both directions. Currently, PASTA uses the ML score as its optimization criterion which is a good score in phylogeny estimation but cannot be proven as a necessary and sufficient criterion to produce an accurate phylogenetic tree. Therefore the integration of multiple application-aware objectives, carefully chosen considering better association to the tree accuracy, into PASTA may potentially have a profound positive impact on its performance. This paper employed four application-aware objectives alongside ML score to develop a multi-objective (MO) framework, namely, PMAO, that leverages PASTA to generate a bunch of high-quality solutions that are considered equivalent in the context of conflicting objectives under consideration. We analyzed this tree-space based on the tree generated by PASTA by experimenting on a popular biological benchmark and found that the tree-space contains significantly better trees than PASTA. 
To help the domain expert further in choosing the most appropriate tree from the PMAO output (containing a relatively large set of high-quality solutions), we incorporated a machine learning approach within the PMAO framework that is capable of generating a smaller set of high-quality solutions. Additionally, we attempted to obtain a single high-quality solution without using any external evidence and found that summarizing the few solutions detected through machine learning can serve this purpose to some extent. \\
%To help the domain expert further in finding the most appropriate tree, we showed that PMAO framework could be assisted by an appropriate machine learning approach to generate a few solutions with at least one having high quality. We further attempted to obtain a single high-quality solution without using any external evidence. We found that summarizing the few solutions detected through machine learning can serve this purpose to some extent. \\
%\textbf{Motivation:} .\\
%\textbf{Results:} .\\
\textbf{Availability:} Our method is freely available at \href{https://github.com/ali-nayeem/pmao}{https://github.com/ali-nayeem/pmao}.\\
%\textbf{Contact:} \href{ali\_nayeem@cse.buet.ac.bd}{ali\_nayeem@cse.buet.ac.bd}\\
\textbf{Supplementary information:} A supplementary file is provided along with the main text.
}
\keywords{Multiple sequence alignment, Phylogenetic tree, Multi-objective optimization}