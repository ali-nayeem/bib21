
\colorlet{lcfree}{green}
\colorlet{lcnorm}{blue}
\colorlet{lccong}{red}

% -------------------------------------------------
% Set up a new layer for the debugging marks, and make sure it is on
% top
\pgfdeclarelayer{marx}
\pgfsetlayers{main,marx}
% A macro for marking coordinates (specific to the coordinate naming
% scheme used here). Swap the following 2 definitions to deactivate
% marks.
\providecommand{\cmark}[2][]{%
  \begin{pgfonlayer}{marx}
    \node [nmark] at (c#2#1) {#2};
  \end{pgfonlayer}{marx}
  } 
\providecommand{\cmark}[2][]{\relax} 
% -------------------------------------------------

%\begin{figure}[h]

%\centering
% -------------------------------------------------
% Start the picture
%\subfloat[BB50016]{
\resizebox{0.44\textwidth}{!}{%
\begin{tikzpicture}[%
    >=triangle 60,              % Nice arrows; your taste may be different
    start chain=going below,    % General flow is top-to-bottom
    node distance=6mm and 60mm, % Global setup of box spacing
    every join/.style={norm},   % Default linetype for connecting boxes
    ]
% ------------------------------------------------- 
% A few box styles 
% <on chain> *and* <on grid> reduce the need for manual relative
% positioning of nodes
\tikzset{
  base/.style={draw, on chain, on grid, align=center, minimum height=4ex},
  proc/.style={base, rectangle, text width=19em},
  proc2/.style={base, rectangle, text width=12em},
  test/.style={base, diamond, aspect=2, text width=6em},
  term/.style={proc, rounded corners},
  term2/.style={proc2, rounded corners},
  merge/.style={base, circle},
  % coord node style is used for placing corners of connecting lines
  coord/.style={coordinate, on chain, on grid, node distance=6mm and 25mm},
  % nmark node style is used for coordinate debugging marks
  nmark/.style={draw, cyan, circle, font={\sffamily\bfseries}},
  % -------------------------------------------------
  % Connector line styles for different parts of the diagram
  norm/.style={->, draw, lcnorm},
  free/.style={->, draw, lcfree},
  cong/.style={->, draw, lccong},
  it/.style={font={\small\itshape}}
}
% -------------------------------------------------
% Start by placing the nodes
%\node [proc, densely dotted, it] (p0) {Randomly initialize base locations};
% Use join to connect a node to the previous one 
\node [term, fill=lcfree!25] {Input: unaligned sequences,\\ \scriptsize{$<W_{SIMG},W_{SIMNG},W_{SOP},W_{GAP},W_{ML}>$}};
\node [proc, join] (p0) {$iA \gets$ compute initial alignment\\$iT \gets$ infer ML tree};
\node [proc, join,  fill=lcfree!25] (p01) {$stocDecom \gets False$};
%\node [proc, join] (p1) {$iT \gets$ estimate ML tree};
\node [proc, join, fill=lcfree!25] (p12) {\scriptsize{$SIMG, SIMNG, SOP, GAP, ML \gets$} calculate 5 scores from ($iA, iT$)};
\node [proc, join, , fill=lcfree!25] (p13) {\scriptsize{$bS \gets SIMG \times W_{SIMG} + SIMNG \times W_{SIMNG} + SOP \times W_{SOP} + GAP \times W_{GAP} + ML \times W_{ML}$}};
\node [proc, join] (p14) {($nA, nT) \gets$ run a PASTA iteration to get a new (alignment, tree) pair from ($iA, iT$)\\{~{\footnotesize\ttfamily{/*$stocDecom$  triggers the stochastic decomposition*/}}} };
\node [proc, join, fill=lcfree!25] (p15) {\scriptsize{$SIMG, SIMNG, SOP, GAP, ML \gets$} calculate 5 scores from ($nA, nT$)};
\node [proc, join, fill=lcfree!25] (p16) {\scriptsize{$nS \gets SIMG \times W_{SIMG} + SIMNG \times W_{SIMNG} + SOP \times W_{SOP} + GAP \times W_{GAP} + ML \times W_{ML}$}};
%\node [proc, join, fill=lccong!25, below=1.7cm of p14] (S1) {Update base locations based on current allocations};
%\node [proc, join, fill=lcfree!25, below=1.6cm of S1] (p41) {Calculate allocations for current bases};
%\node [proc, join=by cong]      {Calculate objective with current base locations and allocations};
\node [test,right=6.3cm of p0] (t1) {$(nS > bS)$?};
% No join for exits from test nodes - connections have more complex
% requirements
% We continue until all the blocks are positioned
\node [proc2] (p2) {($bA, bT) \gets (nA, nT)$\\$bS \gets nS$};
\node [proc2, join,  fill=lcfree!25] (p24) {$stocDecom \gets False$};
\node [proc2,  fill=lcfree!25] (p23) {$stocDecom \gets True$};
\node [merge, join] (p21) {};
\node [proc2, join] (p22) {($iA, iT) \gets (nA, nT)$};
% We position the next block explicitly as the first block in the
% second column.  The chain 'comes along with us'. The distance
% between columns has already been defined, so we don't need to
% specify it.
\node [test, join] (t6) {terminate?}; %, right=8cm of S1
%\node [test, join=by free] (t5) {Accept $S_{new}$?};
%\node [proc, join=by free] (p5) {$S_{candidate} \gets S_{new}$};
%\node [proc, join] (p6) {Maintain $S_{best}$};
%\node [proc, join] (p7) {Clear $SEQ$};
%\node [test, join] (t6) {terminate?};
% Some more nodes specifically positioned (we could have avoided this,
% but try it and you'll see the result is ugly).
\node [term2, join] (p10) {Return ($bA, bT$)};
% -------------------------------------------------
% Now we place the coordinate nodes for the connectors with angles, or
% with annotations. We also mark them for debugging.
%DISABLED BY AHMED
\node [coord, right=2.2cm of t1] (c1)  {}; %\cmark{1}   
\node [coord, left=of t6] (c6)  {}; %\cmark{6}   
%\node [coord, right=of t5]  (c8)  {}; %\cmark{5}  
\node [coord, left=of t6, xshift=-17em]  (cS1)  {}; %\cmark{5}  

% -------------------------------------------------
% A couple of boxes have annotations
%\node [above=0mm of S1, it] {(Location step: Heuristic)~~};
%\node [above=0mm of p41, it] {(Allocation step: Exact) };
% -------------------------------------------------
% All the other connections come out of tests and need annotating
% First, the straight north-south connections. In each case, we first
% draw a path with a (consistently positioned) annotation node, then
% we draw the arrow itself.
\path (t1.south) to node [near start, xshift=1em] {$y$} (p2);
  \draw [*->,lcnorm] (t1.south) -- (p2);
%\path (t5.south) to node [near start, xshift=1em] {$y$} (p5); 
%  \draw [*->,lcfree] (t5.south) -- (p5);
\path (t6.south) to node [near start, xshift=1em] {$y$} (p10); 
  \draw [*->,lcnorm] (t6.south) -- (p10);
% -------------------------------------------------
% Finally, the twisty connectors. Again, we place the annotation
% first, then draw the connector
\path (t1.east) to node [near start, yshift=1em] {$n$} (c1); 
  \draw [*->,lcnorm] (t1.east) -- (c1) |- (p23);
  
\draw [->,lcnorm] (p24.west) -- ++(-4mm,0) |- (p21);


\path (t6.west) to node [yshift=-1em] {$n$} (c6); 
  \draw [*->,lcnorm] (t6.west) -- ++(-16mm,0)  |- (p14.east);

% -------------------------------------------------
% A last flourish which breaks all the rules
%\draw [->,MediumPurple4, dotted, thick, shorten >=1mm]
%  (p2.south) -- ++(5mm,-3mm)  -- ++(27mm,0) 
%  |- node [black, near end, yshift=0.75em, it] {SM to SA} (p4);

\draw [->,lcnorm]
  (p16.south) -- ++(5mm,-3mm)  -- ++(31mm,0) 
  |- node [black, near end, yshift=0.75em, it] {} (t1.west);
  
\end{tikzpicture}
}
%}
%\end{adjustwidth}
%\caption{A high-level workflow of PMAO for one weight vector.}
%\label{fig:flowchart}
%\end{figure}

